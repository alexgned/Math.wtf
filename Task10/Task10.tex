\documentclass[a4paper,portrait,12pt]{article}
\usepackage[english,russian]{babel}
\usepackage[utf8x]{inputenc}
\usepackage{geometry}
\usepackage{xcolor}
\usepackage{hyperref}
\geometry{left=2cm}
\geometry{right=1.5cm}
\geometry{top=1cm}
\geometry{bottom=2cm}
\begin{document}
\begin{center}
\large\textbf{Задание 10 / Task 10 \\ \url{https://math.wtf/course/se_intro2016/task/10}}
\end{center}
\section*{Задание №1}\addcontentsline{toc}{section}{Задание №1}
\textbf{Дано:}\\
\textit{Определим пару (двухэлементный кортеж) как pair = $\lambda$xyf.f x y. Реализовать проекции fst и snd, возвращающие первый и второй элементы пары:}
\begin{itemize}
\item \textit{fst (pair a b) = a}
\item \textit{snd (pair a b) = b}
\end{itemize}
\textbf{Решение:}
\[pair\ a\ b \ {\to}\  {\lambda}f.f a b\]
Для получения первого числа (a) необходимо применить к f комбинатор K = $\lambda$xy.x
\[fst\ = {\lambda}f.f\ K \ {\to}\   {\lambda}f.f({\lambda}xy.x) \]
Для получения второго числа (b) необходимо применить к f комбинатор K* = $\lambda$xy.y
\[snd\ = {\lambda}f.f\ K* \ {\to}\   {\lambda}f.f({\lambda}xy.y) \]
\textbf{Ответ:}
\begin{itemize}
\item fst = $\lambda$f.f($\lambda$xy.x)
\item snd = $\lambda$f.f($\lambda$xy.y)
\end{itemize}
\section*{Задание №2}\addcontentsline{toc}{section}{Задание №2}
\textbf{Дано:}\\
\textit{Выполнить подстановку и $\beta$-преобразование:}
\begin{itemize}
\item \textit{$\lambda$yz.xyw(zx)     [x := $\lambda$y.yw]}
\item \textit{$\lambda$xy.xy($\lambda$x.xy)x  [x := $\lambda$z.z]}
\item \textit{xy($\lambda$xz.xyz)y    [y := xz]}
\end{itemize}
\textbf{Решение:}
\[{\lambda}yz.xyw(zx)\ [x := {\lambda}y.yw] \to {\lambda}yz.({\lambda}y.yw)yw(z({\lambda}y.yw)) \to {\lambda}yz.yww(z({\lambda}y.yw)) \]
\[xy({\lambda}xz.xyz)y\ [y := xz] \to x(xz)({\lambda}x'z'.x'(xz)z')(xz)\]
\textbf{Ответ:}
\begin{itemize}
\item $\lambda$yz.yww(z($\lambda$y.yw))
\item $\lambda$xy.xy($\lambda$x.xy)x \textit{(Без изменений)}
\item x(xz)($\lambda$x'z'.x'(xz)z')(xz)
\end{itemize}
\section*{Задание №3}\addcontentsline{toc}{section}{Задание №3}
\textbf{Дано:}\\
\textit{Показать, что:}
\begin{itemize}
\item \textit{SKK = I}
\item \textit{B = S(KS)K}
\end{itemize}
\textbf{Решение:}
\[S = {\lambda}fgx.fx(gx)\]
\[K = {\lambda}xy.x\]
\[SKK = ({\lambda} x.{\lambda} y.{\lambda} z.xz(yz))({\lambda} x.{\lambda} y.x)({\lambda} x.{\lambda} y.x) \to({\lambda} y.{\lambda} z.({\lambda} x.{\lambda} y.x \, z)(yz))({\lambda} x.{\lambda} y.x) \to \]
 \[  \to {\lambda} z.({\lambda} x.{\lambda} y.x \, z)({\lambda} x.{\lambda} y.x \, z) \to {\lambda} z.({\lambda} y.z)({\lambda} y.z)  \to {\lambda} z.z = I\]
\textbf{\[SKK = I\]}\\
\[B = {\lambda}fgx.f(gx)\]
\[B = {\lambda}fgx.f(gx)fgx \to ({\lambda}xy.x)fx(gx) \to ({\lambda}fgx.fx(gx))(({\lambda}xy.x)f)gx \to \] \[ \to({\lambda}xy.x)({\lambda}fgx.fx(gx))f({\lambda}xy.x)f)gx \to \] \[ \to ({\lambda}fgx.fx(gx))(({\lambda}xy.x)({\lambda}fgx.fx(gx)))({\lambda}xy.x)fgx = S(KS)K \]
\textbf{\[B = S(KS)K\]}
\textbf{Ответ:}
\begin{itemize}
\item \textit{SKK = I}
\item \textit{B = S(KS)K}
\end{itemize}
\end{document}
